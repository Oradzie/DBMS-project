\section{Introduzione}

Il gruppo si pone l'obiettivo di realizzare un sistema informativo che gestisce in modo intelligente una catena di vendita online, con i relativi magazzini, fornitori.

Il focus principale della base di dati è la gestione dei magazzini.

Un utente può registrarsi come cliente nel punto vendita online, inserendo le proprie credenziali e un indirizzo al quale viene recapitato l’ordine.

Effettuando un ordine, questo verrà commissionato ad un magazzino, il quale avrà un amministratore e più magazzinieri.

Arrivato l’ordine, l’amministratore di tale magazzino commissionerà la preparazione di quest’ultimo ai magazzinieri.

L’amministratore del magazzino, annualmente, provvede a determinare l’approvvigionamento per l’anno imminente in base all’andamento delle vendite dell’anno precedente (si tiene traccia quindi di uno storico delle vendite per ogni magazzino).

Inoltre, periodicamente, controlla se è presente la necessità di rifornire il magazzino di determinati prodotti (poiché esauriti).

Tramite l'interfaccia grafica è possibile accedere in qualità di utente online o dipendente (magazziniere o amministratore).
