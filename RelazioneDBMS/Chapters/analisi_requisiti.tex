\section{Analisi dei Requisiti}

\subsection{Requisiti in linguaggio naturale}

La seguente descrizione riporta in linguaggio naturale i requisiti del nostro sistema informativo:

"
Il database dovrà essere in grado di tracciare tutti i prodotti nei magazzini, tenendo conto delle vendite per facilitare il rifornimento dei magazzini in modo preciso e tempestivo.

I clienti che utilizzano il punto vendita online potranno creare un account, inserire le loro informazioni di contatto e di consegna, e poi effettuare ordini. Una volta che un ordine viene effettuato, il sistema lo assegnerà al magazzino appropriato. L'amministratore del magazzino riceverà l'ordine e lo assegnerà ai magazzinieri per la preparazione. 

Nel sistema ci saranno due tipi di utenti interni (dipendenti): i magazzinieri e gli amministratori.
Di ogni dipendente si tiene traccia dell'indirizzo di fatturazione e del suo IBAN.


Gli amministratori avranno la responsabilità di controllare la quantità di prodotti nel magazzino, verificare lo storico delle vendite e commissionare gli ordini giornalieri ricevuti dal punto vendita ai magazzinieri. Saranno anche responsabili di commissionare i rifornimenti del magazzino di riferimento ai fornitori, richiedendo diversi lotti di prodotti.
Gli amministratori dei magazzini hanno anche il compito di determinare l'approvvigionamento annuale basato sulle vendite dell'anno precedente.

I magazzinieri, invece, avranno accesso al database per gestire l'inventario dei prodotti e preparare gli ordini. 
Quando arriva un rifornimento annuale, saranno responsabili dell'aggiornamento del sistema informativo.

Quando un utente online effettua un ordine, questo viene riformulato dal punto vendita in uno o più dettagli ordine, che, in modo intelligente, arriverà al magazzino con la disponibilità più adeguata e da lì, verrà poi preparato il pacco da spedire.
Ogni dettaglio ordine fa riferimento alla versione del prodotto richiesta (ovvero la tipologia di prodotto con le sue specifiche desiderate).
Un determinato prodotto, viene identificato all'interno del magazzino da un numero seriale (codice a barre) e dal suo lotto di appartenenza.

Il magazzino è diviso in settori, scaffali e ripiani.
L'utente può avere diritto ad uno sconto, nel caso ad esempio in cui sia anche dipendente.

L'ordine può presentare, in un determinato periodo, una vendita promozionale, su cui verrà poi applicata una percentuale di sconto."

\newpage
\subsection{Estrazione dei concetti fondamentali}
Individuiamo adesso le parole e le espressioni chiave che ci consentiranno di realizzare uno schema significativo del progetto e di raffinarlo successivamente per ottenere lo schema definitivo. I termini di
rilievo appaiono nel testo con una sottolineatura:

"
Il database dovrà essere in grado di tracciare tutti i prodotti nei magazzini, tenendo conto delle vendite per facilitare il rifornimento dei magazzini in modo preciso e tempestivo.

I clienti che utilizzano il punto vendita online potranno creare un account, inserire le loro informazioni di contatto e di consegna, e poi effettuare ordini. Una volta che un ordine viene effettuato, il sistema lo assegnerà al magazzino appropriato. L'amministratore del magazzino riceverà l'ordine e lo assegnerà ai magazzinieri per la preparazione. 

Nel sistema ci saranno due tipi di utenti interni (\textbf{\underline{dipendenti}}): i magazzinieri e gli amministratori.
Di ogni dipendente si tiene traccia dell'indirizzo di fatturazione e del suo IBAN.

Gli \textbf{\underline{amministratori}} avranno la responsabilità di controllare la quantità di prodotti nel magazzino, verificare lo storico delle vendite e commissionare gli ordini giornalieri ricevuti dal punto vendita ai magazzinieri. Saranno anche responsabili di commissionare i \textbf{\underline{rifornimenti}} del magazzino di riferimento ai \textbf{\underline{fornitori}}, richiedendo diversi \textbf{\underline{lotti}} di prodotti.
Gli amministratori dei magazzini hanno anche il compito di determinare l'approvvigionamento annuale basato sulle vendite dell'anno precedente.

I \textbf{\underline{magazzinieri}}, invece, avranno accesso al database per gestire l'\textbf{\underline{inventario}} dei prodotti e preparare gli ordini. 
Quando arriva un rifornimento annuale, saranno responsabili dell'aggiornamento del sistema informativo.

Quando un \textbf{\underline{utente online}} effettua un \textbf{\underline{ordine}}, questo viene riformulato dal punto vendita in uno o più dettagli ordine, che, in modo intelligente, arriverà al magazzino con la disponibilità più adeguata e da lì, verrà poi preparato il \textbf{\underline{pacco da spedire}}.
Ogni \textbf{\underline{dettaglio ordine}} fa riferimento alla \textbf{\underline{versione del prodotto}} richiesta (ovvero la \textbf{\underline{tipologia di prodotto}} con le sue specifiche desiderate).
Un determinato \textbf{\underline{prodotto}}, viene identificato all'interno del magazzino da un numero seriale (codice a barre) e dal suo lotto di appartenenza.

Il \textbf{\underline{magazzino}} è diviso in \textbf{\underline{settori}}, \textbf{\underline{scaffali}} e \textbf{\underline{ripiani}}.
L'utente può avere diritto ad uno \textbf{\underline{sconto}}, nel caso ad esempio in cui sia anche dipendente.

L'ordine può presentare, in un determinato periodo, una \textbf{\underline{vendita promozionale}}, su cui verrà poi applicata una percentuale di sconto."
\\\\\\
L'azienda tiene a specificare la differenza tra:
\begin{itemize}
    \item Tipologia del prodotto (o categoria), il quale rappresenta il nome del prodotto senza le varie specifiche, ad esempio: "Iphone 13";
    \item Versione prodotto, ovvero il prodotto con la presenza delle specifiche, ad esempio: "Iphone 13 rosso 256gb ecc.";
    \item Prodotto, che rappresenta il singolo prodotto che viene scannerizzato tramite il suo codice a barre (univoco).
\end{itemize}
